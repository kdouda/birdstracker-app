\chapter*{Závěr}
\addcontentsline{toc}{chapter}{Závěr}

Cílem této práce bylo vytvořit offline-capable mobilní aplikaci splňující požadavky ornitologů pro práci v poli. Tento cíl byl splněn vytvořením aplikace ve frameworku React Native, resp. Expo. Aplikace byla vytvořena a otestována autorem na obou hlavních mobilních platformách a zařízeních různých typů a splňuje téměř všechny vydefinované funkční i nefunkční požadavky, z kritických funkčních požadavků všechny. Aplikace kvůli komplikacím způsobené COVID-19 nemohla být včas nasazena, jelikož distribuční platformy nestíhaly zpracovávat požadavky na nové aplikace, nebylo ani možné distribuovat testovací verzi vybrané sekci potenciálních uživatelů. 

Vývoj aplikace bude pokračovat a stane se užitečnou pomůckou při výzkumných projektech klientů platformy Anitra. Ostrý provoz aplikace se očekává po červnu 2020, kdy se do aplikace doimplementují některé chybějící moduly a aplikace se výrazně zoptimalizuje. Klientům bude dodána přes internetové obchody mobilních aplikací Android Google Play a iOS App Store, o vydání aplikace se dozví z newsletteru platformy Anitra či z webové stránky. Do aplikace je nutno doimplementovat zadávání bodů zájmu a zlepšit zobrazování a nahrávání tras, aby fungovalo na pozadí aplikace a minimalizovalo spotřebu energie.

Tato práce se věnovala vývoji mobilní aplikaci pro podporu práce ornitologů v poli. Práce popsala problémovou oblast, kde se autor snažil nastínit hrubý vývoj ornitologie a zvířecí telemetrie, která je pro tuto práci klíčovou. V následující kapitole byla rozebrána existující řešení vícero různých problémů, se kterými je nutno se potýkat při práci v terénu i v přípravě k těmto činnostem. Následující kapitoly se věnovaly běžnému postupu při návrhu a implementaci malých softwarových aplikací. Prvně byly identifikováni stakeholdeři, sesbírány požadavky a z nich byly sestaveny funkční i nefunkční požadavky na aplikaci. V následující kapitole byla vybrána technologie pro řešení aplikace. Kapitola Návrh se věnovala návrhu uživatelského rozhraní i softwarové architektury aplikace, v kapitole implementace se tyto navrhnuté modely realizovaly. V poslední kapitole nasazení a testování byla aplikace testována na vhodných zařízeních a aplikace v testu obstála.