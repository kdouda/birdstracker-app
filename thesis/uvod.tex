\chapter*{Úvod}
\addcontentsline{toc}{chapter}{Úvod}

S rozvojem IoT technologií dochází ke stále další a další miniaturizaci autonomních off-the-grid zařízení. Jedním z mnoha oborů, které z tohoto vývoje těží, je zoologie. Obory zoologie zabývající se výzkumem migrací, studiem životních cyklů, ochrany a výzkum vlivu lidské činnosti na zvířata díky tomuto vývoji využívají stále dostupnější trackovací zařízení nasazovaná na zvířata. Nasazená zařízení komunikují především pomocí GSM technologií a sbírají telemetrická data o životních funkcích zvířete, pozici, vzdálenosti k zájmovým bodům a podobně. Zoologové tyto údaje průběžně kontrolují a mohou využívat výpočetní techniku pro asistenci ve výzkumných i konzervačních programech.

Ačkoliv výpočetní technologie se rozvíjí rychle, díky netechnické povaze zoologie lze konstantovat, že specializovaný software pro tuto doménu zastarává jak z technických hledisek, tak z hledisek uživatelského dojmu. Specializovaný software se taktéž často omezuje pouze na výseč určité funkcionality, např. nástroje pro zpracování statistických dat, korekci údajů z trackerů a dalších podpůrných procesů pro velká data. Pro většinu běžných úkonů jako je evidence údajů o sledovaných zvířatech nebo jejich pozice se tedy nadále používají např. tabulkové procesory, obecné GISové nástroje a nestrukturovaný ruční zápis dat.

Tyto technologie nejsou efektivní pro synchronizaci a zobrazení dat pro práci v terénu, z čehož plyne neefektivní využití času odborných pracovníků, kteří musí trávit čas přenosem dat. Pro úkony v terénu je tento proces výrazně komplikovanější nedostupností signálu mimo obce v hustých porostech. Komplikací je taktéž s sebou nosit zařízení větší, než je mobilní telefon. Mobilní aplikace není nutně jediným způsobem řešení daného problému, ale autor práce toto řešení považuje za nejdostupnější. Jiná řešení již byla nastíněna, jedná se např. o zmiňované způsoby nestrukturovaného zápisu, využití GPS navigací s předprogramovanými lokacemi a další. Mobilní aplikace dokáže tyto způsoby nahradit jedním centrálním řešením, které je navíc do budoucna rozšiřitelné bez jakýchkoliv nákladů na straně uživatele. 

\section*{Cíle práce a způsob jejího dosažení}

Cílem této práce je vytvoření mobilní aplikace pro sledování a kontrolu divokých zvířat v terénu, konkrétněji ptáků. Primární funkcí aplikace musí být efektivní zobrazení posledních pozic uživatelem sledovaných zvířat osazených GPS-GSM trackery v mapě a zobrazení pozice uživatele. Druhotnou funkcí je zobrazování a zadávání metainformací k GPS-GSM trackerům, resp. konkrétním zvířatům, které tyto trackery nosí. Pro aplikaci je kritická možnost fungování bez internetového připojení, kterou současné řešení nepodporuje. Současná řešení taktéž nejsou vhodná pro použití na mobilu z hlediska použitelnosti. Aplikace je cílená především pro ornitology, kteří využívají možností zvířecí telemetrie pro práci v terénu.

Cíl práce bude dosažen aplikací metodik softwarového inženýrství použitím vodopádového modelu vývoje aplikací bez fáze údržby. Každé fázi vodopádového modelu odpovídá kapitola v textu práce. Zvolený způsob implementace řešení je multiplatformní aplikace vyvíjená v prostředí React Native s možností práce offline. React Native umožňuje vytváření mobilních aplikací pro platformy Android i iOS bez nutnosti psát dvě separátní aplikace. React Native je souborem JS knihoven postavených nad frontendovým frameworkem React od společnosti Facebook. Pro zrychlení vývoje bez nutnosti podrobného testování aplikací na obou platformách byla použita nástavba Expo, která dále abstrahuje od platformně závislého kódu. Zdrojem dat je ornitologická platforma Anitra, která mimo funkce datového úložiště podporuje sdílení dat a udržování metainformací o zvířatech.

\section*{Předpoklady a omezení práce}
Text této práce lze považovat za užitečný pro popis procesu vývoje malých multiplatformních mobilních aplikací s podporou off-line režimu. Text samotný nelze považovat za návod pro vytvoření mobilní aplikace, ale z textu samotného lze vybrat určité koncepty, které jsou přenositelné i do dalších mobilních aplikací, které vyžadují práci off-line. Od čtenáře se čeká elementární znalost programovacího jazyku JavaScript, či alespoň podobné syntaxe (např. jazyk Java, C++) a základních konceptů vývoje softwaru. Práce se základy syntaxe jazyků nezabývá. Základ práce s použitými knihovnami je v textu uveden, cílem práce ale opět není sloužit jako návod.  Text práce samotný využívá metodik vědeckých prací, konkrétněji metodiku rešerše pro rešerši existujících poznatků o vývoji mobilních aplikací vyplývajících z odborných prací. 

Toto téma je řešeno z důvodu absence efektivního řešení problémů, se kterými se ornitologové běžně potýkají. Nevýhody jednotlivých současných řešení jsou uvedeny v textu práce.

\section*{Struktura práce}
Tento dokument se skládá ze sedmi kapitol. První kapitola ve zkratce popisuje problémovou oblast, současný stav a základní koncepty nutné pro pochopení zbytku práce související s kontextem ornitologie a telemetrických zařízení pro ornitologii. V druhé kapitole se objasní stávající řešení využívané k docílení určitých potřeb ornitologů, což objasní následující kapitoly a důvod, proč existující řešení nejsou vhodná a proč došlo k rozhodnutí vyvinout mobilní aplikaci. Samotným vývojovým procesem se tento dokument zabývá od kapitoly Analýza požadavků, ve které se uvedou funkční požadavky na mobilní aplikaci tohoto typu. Kapitola technologie popisuje technologie, které pro řešení tohoto problému šlo využít, jak a jakým způsobem byly vybrány a základní informace o jednotlivých částech technologií. Kapitola návrh popíše návrh aplikačních architektury a návrh uživatelského rozhraní. Kapitola implementace popíše proces vytváření samotné aplikace a věnuje se specifickému řešení problémů z předchozí kapitoly. Poslední kapitolou je testování a nasazení, kde je popsán způsob, jakým byla aplikace testována a publikována. Na závěr je zhodnocení splnění cílů práce a možnost dalšího rozvoje této práce.

\section*{Rešerše}

Cílem této podkapitoly je popsat aktuální odborné poznatky na téma vývoj mobilních aplikací, multiplatformních možností vývoje těchto aplikací, úskalí návrhů mobilních aplikací a z hlediska funkcionality využití polohovacích funkcí mobilních zařízení a synchronizaci dat vůči vzdálenému serveru.

Zdroje pro jednotlivé odborné texty byly citační rejstříky:

\begin{itemize}
	\item Theses.cz,
	\item VŠKP VŠE,
	\item Google Scholar.
\end{itemize}

Hledaná klíčová slova byla: "mobilní aplikace", "multiplatformní mobilní aplikace", "mobilní aplikace poloha", "multiplatformní vývoj". Cílem bylo vybrat zdroje, které se věnují různým aspektům problematiky, kterou tato práce bude řešit.

Problematikou návrhu a realizace geolokačních služeb v mobilních aplikacích se zabýval Eduard  \textcite{Bakes2018thesis} ve své diplomové práci. Tato práce dopodrobna popisuje jednotlivé způsoby, jaké telefonní zařízení využívají pro stanovení lokality, způsoby reprezentace této polohy, principy jejich fungování a tyto způsoby srovnává. Z tohoto srovnání vyšly nejpřesněji GNSS služby. Autor se dále věnuje implementaci základních mapových interakcí pro OS Android. Hlavním poznatkem autora z této části je komplikovanost vývoje i základních mapových interakcí pro obě platformy zvlášť a je dle jeho názoru lepší využít řešení, které již od jednotlivých rozdílů abstrahuje.

% https://is.muni.cz/th/u09n4/multiplatformni-mobilni-aplikace-text.pdf

Návrhem a vývojem multiplatformní mobilní aplikace, sloužící jako klient webového prostředí, se zabývá Petr \textcite{Domkar2018} ve své diplomové práci. Aplikace nastiňuje možnosti řešení těchto mobilních aplikací, jako je například progresivní webová aplikace, hybridní aplikace, Xamarin a React Native a NativeScript (poslední tři zahrnuty pod termín multiplatformní). Jednotlivé přístupy srovnává a usuzuje, že dle stanovených kritérií se vývoj multiplatformních aplikací zdá nejlepší. V práci následně popisuje proces návrhu a vývoje mobilní aplikace za použití multiplatformní technologie NativeScript. Práce je strukturálně i tvořenou aplikací podobná této, i přes rozdíly ve zvolené technologii. Autor práce taktéž zvolil stejný způsob řešení práce využitím JS frameworku pro tvorbu mobilních aplikací.

Analýzou použitelnosti jedné z často používaných technologií pro vývoj multiplatformních mobilních aplikací se zabýval Jakub \textcite{Menda2018thesis} ve své diplomové práci. Práce o frameworku React Native se zabývá vývojem modelové aplikace a následného posouzení frameworku dle normy ISO/IEC 9126. Autor taktéž důsledně popsal úskalí vývoje za pomocí frameworku React Native. Pro autora této práce byla práce přínosnou pro posouzení jakosti, tedy kvality vytvořeného softwaru. Vztažením tohoto modelu jakosti i na možnosti frameworku samotného lze z prezentovaných argumentů brát jako ujištění, že tato technologie je vhodnou pro vývoj kvalitních mobilních aplikací.

Vývojem aplikace podobného rozsahu s porovnatelnými funkcionalitami se zabýval Zdeněk  \textcite{Tomka2018thesis} ve své bakalářské práci, ve které se věnuje tvorbě aplikace pomocí frameworku React Native. Tato práce je zajímavá z hlediska komunikace se vzdáleným serverem, uživatelské vstupy (generované z mapy) se vkládají do vzdáleného uložiště. Práce je taktéž zajímavá provedením uživatelských testů, které by autor této práce rád provedl obdobným způsobem.

Konferenční příspěvek sepsaný \textcite{caseres2019study} se věnuje nefunknčním požadavkům v mobilních aplikací a srovnává šest technologií pro řešení mobilních aplikací, od nativních aplikací po hybridní aplikace. Jednotlivé způsoby srovnává z hledisek výkonu, využití energie a využití místa na disku, které byly identifikovány jako klíčové nefunkční požadavky pro mobilní aplikace. Hlavním přínosem této práce je vytipování těchto nefunkčních požadavků a konkrétní čísla, od kterých se může rozvíjet volba vhodné technologie pro řešení nové mobilní aplikace.
