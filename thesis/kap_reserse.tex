\section*{Rešerše}

Cílem této kapitoly je popsat aktuální odborné poznatky na téma vývoj mobilních aplikací, multiplatformních možností vývoje těchto aplikací, úskalí návrhů mobilních aplikací a z hlediska funkcionality využití polohovacích funkcí mobilních zařízení a synchronizaci dat vůči vzdálenému serveru.

Zdroje pro jednotlivé odborné texty byly citační rejstříky:

\begin{itemize}
	\item Theses.cz,
	\item VŠKP VŠE,
	\item Google Scholar.
\end{itemize}

Hledaná klíčová slova byla: "mobilní aplikace", "multiplatformní mobilní aplikace", "mobilní aplikace poloha", "multiplatformní vývoj". Cílem bylo vybrat zdroje, které se věnují různým aspektům problematiky, kterou tato práce bude řešit.

Problematikou návrhu a realizace geolokačních služeb v mobilních aplikacích se zabýval Eduard  \textcite{Bakes2018thesis} ve své diplomové práci. Tato práce dopodrobna popisuje jednotlivé způsoby, jaké telefonní zařízení využívají pro stanovení lokality, způsoby reprezentace této polohy, principy jejich fungování a tyto způsoby srovnává. Z tohoto srovnání vyšly nejpřesněji GNSS služby. Autor se dále věnuje implementaci základních mapových interakcí pro OS Android. Hlavním poznatkem autora z této části je komplikovanost vývoje i základních mapových interakcí pro obě platformy zvlášť a je dle jeho názoru lepší využít řešení, které již od jednotlivých rozdílů abstrahuje.

% https://is.muni.cz/th/u09n4/multiplatformni-mobilni-aplikace-text.pdf

Návrhem a vývojem multiplatformní mobilní aplikace, sloužící jako klient webového prostředí, se zabývá Petr \textcite{Domkar2018} ve své diplomové práci. Aplikace nastiňuje možnosti řešení těchto mobilních aplikací, jako je například progresivní webová aplikace, hybridní aplikace, Xamarin a React Native a NativeScript (poslední tři zahrnuty pod termín multiplatformní). Jednotlivé přístupy srovnává a usuzuje, že dle stanovených kritérií se vývoj multiplatformních aplikací zdá nejlepší. V práci následně popisuje proces návrhu a vývoje mobilní aplikace za použití multiplatformní technologie NativeScript. Práce je strukturálně i tvořenou aplikací podobná této, i přes rozdíly ve zvolené technologii. Autor práce taktéž zvolil stejný způsob řešení práce využitím JS frameworku pro tvorbu mobilních aplikací.

Analýzou použitelnosti jedné z často používaných technologií pro vývoj multiplatformních mobilních aplikací se zabýval Jakub \textcite{Menda2018thesis} ve své diplomové práci. Práce o frameworku React Native se zabývá vývojem modelové aplikace a následného posouzení frameworku dle normy ISO/IEC 9126. Autor taktéž důsledně popsal úskalí vývoje za pomocí frameworku React Native. Pro autora této práce byla práce přínosnou pro posouzení jakosti, tedy kvality vytvořeného softwaru. Vztáhnutím tohoto modelu jakosti i na možnosti frameworku samotného lze z prezentovaných argumentů brát jako ujištění, že tato technologie je vhodnou pro vývoj kvalitních mobilních aplikací.

%Podobnou problematikou se zabýval i Zdeněk  \textcite{Tomka2018thesis} ve své bakalářské práci, ve které se věnuje tvorbě aplikace pomocí frameworku Rect Native. Autor této práce 
